\documentclass[]{article}
\usepackage{graphicx}
\graphicspath{{images/}}
\usepackage{geometry}
 \geometry{
 a4paper,
 total={210mm,297mm},
 left=30mm,
 right=30mm,
 top=20mm,
 bottom=20mm,
 }
\title{Solving Travelling Salesman Problem using Simple Genetic Algorithm (SGA)}
\author{Kshitij Goel (13AE10013)}

\begin{document}

\maketitle
\begin{center}
\centering
\textbf{Term Paper, Intelligent Control (EE60028)}
\end{center}
\begin{abstract}
Travelling Salesman Problem involves determination of optimal path between a set of points such that if a salesman walks through all of them only once.
The problem is very non-linear in nature and can't be efficiently solved using conventional optimization techniques. Hence, here the problem is attempted
to be solved by a heuristic non-conventional optimization technique called Genetic Algorithm. Precisely, here only Simple Genetic Algorithm (SGA) is under consideration.
Positions of cities are assigned randomly. Optimal path plot, plot showing position of  cities, distance (cost) matrix and Cost v/s time plots are obtained. 
\end{abstract}

\section{Travelling Salesman Problem}
Inputs to the problem :
\begin{itemize}
\item{No. of points to be visited by the salesman (it can't cross the same points again)} 
\item{No. of iterations before the algorithm convergers}
\end{itemize}
Outputs expected :
\begin{itemize}
\item{Plot showing the randomly generated position of cities}
\item{Plot showing the Optimal Path along these cities}
\end{itemize}
\section{Subroutines}
\subsection{\tt{Input/Output}}
I/O is implemented as a \tt{MATLAB} structure and has the fields :
\begin{itemize}
\item 
\end{itemize}
\begin{figure}
\centering
\textbf{Sample Random Input of positions of $50$ Cities}\par \medskip
\includegraphics[scale = 0.5]{images/city_pos}
\caption{Random (x,y) allocated to $50$ cities}
\end{figure}

\end{document}
